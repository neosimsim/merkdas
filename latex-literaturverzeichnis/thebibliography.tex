% The MIT License (MIT)
% 
% Copyright (c) 2015 Alexander Ben Nasrallah
% 
% Permission is hereby granted, free of charge, to any person obtaining a copy
% of this software and associated documentation files (the "Software"), to deal
% in the Software without restriction, including without limitation the rights
% to use, copy, modify, merge, publish, distribute, sublicense, and/or sell
% copies of the Software, and to permit persons to whom the Software is
% furnished to do so, subject to the following conditions:
% 
% The above copyright notice and this permission notice shall be included in all
% copies or substantial portions of the Software.
% 
% THE SOFTWARE IS PROVIDED "AS IS", WITHOUT WARRANTY OF ANY KIND, EXPRESS OR
% IMPLIED, INCLUDING BUT NOT LIMITED TO THE WARRANTIES OF MERCHANTABILITY,
% FITNESS FOR A PARTICULAR PURPOSE AND NONINFRINGEMENT. IN NO EVENT SHALL THE
% AUTHORS OR COPYRIGHT HOLDERS BE LIABLE FOR ANY CLAIM, DAMAGES OR OTHER
% LIABILITY, WHETHER IN AN ACTION OF CONTRACT, TORT OR OTHERWISE, ARISING FROM,
% OUT OF OR IN CONNECTION WITH THE SOFTWARE OR THE USE OR OTHER DEALINGS IN THE
% SOFTWARE.

\documentclass[a4paper]{article}

\usepackage[utf8]{inputenc}
\usepackage[ngerman]{babel}
\usepackage{listings}
\usepackage[colorlinks=true, linktoc=page]{hyperref}
\usepackage[]{xcolor}

\lstdefinestyle{LaTeX}{
	keywordstyle=\bfseries\color{green!40!black},
	texcsstyle=*\bfseries\color{red!80!black},
	backgroundcolor=\color{white},
	frame=single,
}

\lstset{
	language=[LaTeX]TeX,
	morekeywords={thebibliography,lamport94},
	tabsize=4,
	style=LaTeX
}

\title{thebibliography}
\author{Alexander Ben Nasrallah}

\begin{document}
\maketitle

\begin{abstract}
Es gibt verschiedene Varianten ein Literaturverzeichnis mit \LaTeX{} zu
erstellen. Hier erläutern wir die Verwendung der eingebetteten \verb+thebibliography+
Umgebung.
\end{abstract}

\section{thebibliography}
\LaTeX{} liefert bereits eine Umgebung mit, \verb+thebibliography+. In dieser
Umgebung werden alle Referenzen aufgelistet.

\begin{lstlisting}[]
\begin{thebibliography}{9}
	\bibitem{lamport94}
		Leslie Lamport,
		\emph{\LaTeX: a document preparation system},
		Addison Wesley, Massachusetts,
		2nd edition,
		1994.
	\bibitem{lamport95}
		Leslie Lamport,
		\emph{\LaTeX: a document preparation system},
		Addison Wesley, Massachusetts,
		3nd edition,
		1995.
\end{thebibliography}
\end{lstlisting}

\begin{thebibliography}{9}
	\bibitem{lamport94}	% Der Parameter ist der Referenzschlüssel, muss also eindeutig sein.
		Leslie Lamport,
		\emph{\LaTeX: a document preparation system},
		Addison Wesley, Massachusetts,
		2nd edition,
		1994.
	\bibitem{lamport95}
		Leslie Lamport,
		\emph{\LaTeX: a document preparation system},
		Addison Wesley, Massachusetts,
		3nd edition,
		1995.
\end{thebibliography}

Links verweist man mit \verb+\cite{<item name>}+: \cite{lamport94}.


\url{http://en.wikibooks.org/wiki/LaTeX/Bibliography_Management}
\end{document}


