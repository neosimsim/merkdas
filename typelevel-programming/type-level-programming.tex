\documentclass[ignorenonframetext, aspectratio=169]{beamer}

\definecolor{color-primary}{HTML}{C90000}

\usepackage{neosimsim}
\setsansfont{Roboto}
\beamertemplatenavigationsymbolsempty{}
\logo{\includegraphics[height=6em,trim={0 2em 0 0}]{logo}}
\usepackage{pgfpages}
\setbeameroption{show notes} % on second screen}
\setbeamertemplate{itemize items}{»}
\setbeamercolor{titlelike}{fg=color-primary}
\setbeamercolor{block title}{fg=color-primary}
\setbeamercolor{section in toc}{fg=black}
\setbeamercolor{itemize item}{fg=color-primary}
\setbeamercolor{itemize subitem}{fg=color-primary}
\setbeamercolor{itemize subsubitem}{fg=color-primary}
\setbeamercolor{enumerate item}{fg=color-primary}
\setbeamercolor{enumerate subitem}{fg=color-primary}
\setbeamercolor{enumerate subsubitem}{fg=color-primary}
\setbeamercolor{description item}{fg=color-primary}

\author{Alexander Ben Nasrallah}
\institute{SerNet GmbH}
\title{\textrm{Typelevel Programming}}
\subtitle{An Introduction}

% https://tex.stackexchange.com/questions/232168/normal-text-is-invisible-when-using-beamer-with-notes-and-xelatex
\makeatletter
\def\beamer@framenotesbegin{% at beginning of slide
     \usebeamercolor[fg]{normal text}
      \gdef\beamer@noteitems{}%
      \gdef\beamer@notes{}%
}
\makeatother

\begin{document}
\frame{\titlepage}

\begin{frame}
	\tableofcontents[hideallsubsections]
\end{frame}

\section{Introduction}
\frame{\sectionpage}

\begin{frame}{Goals, Ideas, Concepts etc.}
If it compiles, it works.

Don't make illegate states representable.

Let the compiler do the work.
\note{Each of these could be sections.}
\end{frame}

\subsection{Agda}
\begin{frame}[allowframebreaks, fragile]{Agda}
	\lstinputlisting{lambda.agda}
\end{frame}

I'm not in a frame

\section{Next}
\frame{\sectionpage}

\subsection{…}
\frame{\sectionpage}


\begin{frame}{Test some Elements}
	Hallo

	-

	--

	---

	Welt
	\begin{block}{English}
		Hello

		World
	\end{block}
\end{frame}

\begin{frame}
	\frametitle{`Hidden higher-order concepts?'}
	\begin{itemize}[<+->]
	\item The truths of arithmetic which are independent of PA in some 
	sense themselves `{contain} essentially {\color{blue}{hidden higher-order}},
	 or infinitary, concepts'???
	\item `Truths in the language of arithmetic which \ldots
		\begin{itemize}[<+->]
		\item The truths of arithmetic which are independent of PA in some 
		sense themselves `{contain} essentially {\color{blue}{hidden higher-order}},
		 or infinitary, concepts'???
		\item `Truths in the language of arithmetic which \ldots
		\item	That suggests stronger version of Isaacson's thesis. 
		\end{itemize}
	\item	That suggests stronger version of Isaacson's thesis. 
	\end{itemize}
\end{frame}

\begin{frame}
	\frametitle{`Hidden higher-order concepts?'}
	\begin{enumerate}[<+->]
	\item The truths of arithmetic which are independent of PA in some 
	sense themselves `{contain} essentially {\color{blue}{hidden higher-order}},
	 or infinitary, concepts'???
	\item `Truths in the language of arithmetic which \ldots
		\begin{enumerate}[<+->]
		\item The truths of arithmetic which are independent of PA in some 
		sense themselves `{contain} essentially {\color{blue}{hidden higher-order}},
		 or infinitary, concepts'???
		\item `Truths in the language of arithmetic which \ldots
		\item	That suggests stronger version of Isaacson's thesis. 
		\end{enumerate}
	\item	That suggests stronger version of Isaacson's thesis. 
	\end{enumerate}
\end{frame}


\end{document}
